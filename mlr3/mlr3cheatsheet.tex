\documentclass{beamer}

\usepackage[orientation=landscape,size=a0,scale=1.4,debug]{beamerposter}
\mode<presentation>{\usetheme{mlr}}

\usepackage[utf8]{inputenc} % UTF-8
\usepackage[english]{babel} % Language
\usepackage{hyperref} % Hyperlinks
\usepackage{ragged2e} % Text position
\usepackage[export]{adjustbox} % Image position

\title{mlr3 :\,: CHEAT SHEET} % Package title in header, \, adds thin space between ::
\newcommand{\packagedescription}{ % Package description in header
	The \textbf{mlr3} package provides a framework for classification, regression and other machine learning tasks.
}

\newlength{\columnheight} % Adjust depending on header height
\setlength{\columnheight}{84cm} 

\begin{document}
\begin{frame}[fragile]{}
\begin{columns}
	\begin{column}{.31\textwidth}
		\begin{beamercolorbox}[center]{postercolumn}
			\begin{minipage}{.98\textwidth}
				\parbox[t][\columnheight]{\textwidth}{				
					\begin{myblock}{Task}
						Tasks objects store data (\textit{backend}) and additional meta-data for machine-learning problems.
						\\[2\baselineskip]
						task = \textbf{TaskClassif}\$new(backend, target) - The \textit{target} is a label with only few distinct values. 
						\\[\baselineskip]
						task = \textbf{TaskRegr}\$newt(backend, target) - The \textit{target} is a numeric quantity.
						\\[\baselineskip]
						task\$\textbf{data}() - Retrieves stored data.
						\\[\baselineskip]
						task\$\textbf{select}(cols) -  Subsets the task based on feature names (\textit{cols}).
						\\[\baselineskip]
						task\$\textbf{filter}(rows) - Subsets the task based on row ids (\textit{rows}).
					\end{myblock}
					\begin{myblock}{Learner}
						Learner objects provide a unified interface to machine-learning algorithms.
						\\[2\baselineskip]
						learner = mlr\_learners\$\textbf{get}(id) - Extracts a specific learner by \textit{id}.
						\\[\baselineskip]
						learner\$\textbf{param\_set} - Returns a description of hyperparameter settings.
					\end{myblock}\vfill
				}
			\end{minipage}
		\end{beamercolorbox}
	\end{column}
	\begin{column}{.31\textwidth}
		\begin{beamercolorbox}[center]{postercolumn}
			\begin{minipage}{.98\textwidth}
				\parbox[t][\columnheight]{\textwidth}{
					\begin{myblock}{Train \& Predict}
						Training fits a model to a given task. 
						\\[2\baselineskip]
						train\_set = sample(task\$nrow, 0.8 * task\$nrow) - Get train set.
						\\[\baselineskip]
						test\_set = setdiff(seq\_len(task\$nrow), train\_set) - Get test set.
						\\[\baselineskip]
						learner\$\textbf{train}(task, row\_ids = train\_set) - Train learner with the \textit{train\_set}.
						\\[\baselineskip]
						prediction = learner\$\textbf{predict}(task, row\_ids = test\_set) - Predict \textit{test\_set} with model.
						\\[\baselineskip]
						measure = mlr\_measures\$\textbf{get}(key) - Get measure by \textit{key}.
						\\[\baselineskip]
						prediction\$\textbf{score}(measure) - Access performance with \textit{measure}.
					\end{myblock}
					\begin{myblock}{Resampling}
						Resampling is used to assess the performance of a learning algorithm.
						\\[2\baselineskip]
						\begin{minipage}{\textwidth}
							\begin{columns}[T]
								\begin{column}{0.2\textwidth}
									\includegraphics[width=\textwidth]{img/cross_validation.png}
								\end{column}
								\begin{column}{0.8\textwidth}
										resampling = mlr\_resamplings\$\textbf{get}(key) - Get resampling strategy by \textit{key}. 
								\end{column}
							\end{columns}
						\end{minipage}
						\\[\baselineskip]
						resampling\$\textbf{instantiate}(task) - Apply splitting on \textit{task}.
						\\[\baselineskip]
						rr = \textbf{resample}(task, learner, resampling) - Executes resampling.
						\\[\baselineskip]
						rr\$\textbf{performance}(measure) - Extract performance of individual resampling iterations with \textit{measure}.
						
					\end{myblock}\vfill
				}
			\end{minipage}
		\end{beamercolorbox}
	\end{column}
	\begin{column}{.31\textwidth}
		\begin{beamercolorbox}[center]{postercolumn}
			\begin{minipage}{.98\textwidth}
				\parbox[t][\columnheight]{\textwidth}{
					\begin{myblock}{Benchmarking}
						Benchmarking is used to compare the performance of different learners on multiple task and/or different resampling schemes.
					\end{myblock}
					\begin{myblock}{Packages}
						\href{https://github.com/mlr-org/mlr3pipelines}{mlr3pipelines} - Dataflow Programming\\
						\href{https://github.com/mlr-org/mlr3tuning}{mlr3tuning} - Tuning Methods\\
						\href{https://github.com/mlr-org/mlr3filters}{mlr3filters} - Feature Selection Filters\\ 
						\href{https://github.com/mlr-org/mlr3survival}{mlr3survival} - Survival analysis\\
						\href{https://github.com/mlr-org/mlr3fswrap}{mlr3fswrap} - Wrapper feature selection\\
						\href{https://github.com/mlr-org/mlr3learners}{mlr3learners} - Recommended learners\\
						\href{https://github.com/mlr-org/mlr3viz}{mlr3viz} - Visualizations \\
						\href{https://github.com/mlr-org/mlr3ordinal}{mlr3ordinal} - Ordinal Regression 
					\end{myblock}
					\begin{myblock}{Resources}
						\href{https://mlr3book.mlr-org.com/index.html}{mlr3book}\\
						\href{https://github.com/mlr-org}{mlr-org on GitHub}
					\end{myblock}\vfill
				}
			\end{minipage}
		\end{beamercolorbox}
	\end{column}
\end{columns}
\end{frame}
\end{document}
