\documentclass{beamer}

\usepackage[orientation=landscape,size=a0,scale=1.4,debug]{beamerposter}
\mode<presentation>{\usetheme{mlr}}

\usepackage[sfdefault]{roboto}
\usepackage{roboto-mono}
\usepackage[T1]{fontenc}
\usepackage[utf8]{inputenc} % UTF-8
\usepackage[english]{babel} % Language
\usepackage{hyperref} % Hyperlinks
\usepackage{ragged2e} % Text position
\usepackage[export]{adjustbox} % Image position
\usepackage[most]{tcolorbox}
\usepackage{listings} % for R code
\lstset{language=R,
    basicstyle=\small\ttfamily,
    stringstyle=\color{DarkGreen},
    otherkeywords={0,1,2,3,4,5,6,7,8,9},
    morekeywords={TRUE,FALSE},
    deletekeywords={data,frame,length,as,character},
    keywordstyle=\color{blue},
    commentstyle=\color{DarkGreen}
}
\hypersetup{
    hyperfootnotes=false,
    colorlinks=true,
	linktocpage=true,
	pdfauthor={mlr-org team},
    %linkcolor=[RGB]{3,99,142}, % mlr blue
    urlcolor=[RGB]{231,138,69}
}


\title{mlr3 :\,: CHEAT SHEET} % Package title in header, \, adds thin space between ::
\newcommand{\packagedescription}{ % Package description in header
	The \textbf{mlr3} package provides a framework for classification, regression and other machine learning tasks.
}

\newlength{\columnheight} % Adjust depending on header height
\setlength{\columnheight}{84cm}

\newtcolorbox{codebox}{%
	sharp corners,
	leftrule=0pt,
	rightrule=0pt,
	toprule=0pt,
	bottomrule=0pt,
	fontupper=\robotomono\small,
	hbox}

\newtcolorbox{codeboxmultiline}[1][]{%
	sharp corners,
	leftrule=0pt,
	rightrule=0pt,
	toprule=0pt,
	bottomrule=0pt,
	fontupper=\robotomono\small,
	#1}

\newtcolorbox{codeboxinline}{%
	sharp corners,
	leftrule=0pt,
	rightrule=0pt,
	toprule=0pt,
	bottomrule=0pt,
	hbox,
	nobeforeafter,
	fontupper=\robotomono\small,
	tcbox raise base}

\newcommand{\codeinline}[1]{\begin{codeboxinline}#1\end{codeboxinline}}

\begin{document}
\begin{frame}[fragile]{}
	\begin{columns}
		\begin{column}{.245\textwidth}
			\begin{beamercolorbox}[center]{postercolumn}
				\begin{minipage}{.98\textwidth}
					\parbox[t][\columnheight]{\textwidth}{
						\begin{myblock}{Intro}
							The \textbf{mlr3} package builds on R6 classes and provides the essential building
							blocks of a machine learning workflow.
						\end{myblock}
						\begin{myblock}{Dictionaries}
							Content
						\end{myblock}
					}
				\end{minipage}
			\end{beamercolorbox}
		\end{column}
		\begin{column}{.245\textwidth}
			\begin{beamercolorbox}[center]{postercolumn}
				\begin{minipage}{.98\textwidth}
					\parbox[t][\columnheight]{\textwidth}{
					\begin{myblock}{Task}
						Stores observation and columns (\codeinline{backend}) and additional
						meta-data about the dataset.
						\vspace{1em}
						\\
						\begin{codebox}
							task = \textbf{TaskClassif}\$new(backend, target)
						\end{codebox}
						For classification, \codeinline{target} needs to be a character vector representing class labels.
						\\
						\begin{codebox}
							task = \textbf{TaskRegr}\$new(backend, target)
						\end{codebox}
						For regression, \codeinline{target} needs to be a numeric vector.
						\\
						\begin{codebox}
							task = \textbf{tsk}(.key)
						\end{codebox}
						Built-in tasks can be queried by \codeinline{.key}.
						See \codeinline{mlr\_tasks} for a list of available tasks.
						\vspace{1em}
						\\
						\begin{codebox}
							task\$\textbf{select}(cols)
						\end{codebox}
						Subsets the task based on feature names.
						\\
						\begin{codebox}
							task\$\textbf{filter}(rows)
						\end{codebox}
						Subsets the task based on row ids.
					\end{myblock}
					\vfill
					}
				\end{minipage}
			\end{beamercolorbox}
		\end{column}
		\begin{column}{.245\textwidth}
			\begin{beamercolorbox}[center]{postercolumn}
				\begin{minipage}{.98\textwidth}
					\parbox[t][\columnheight]{\textwidth}{
						\begin{myblock}{Learner}
						Provides a unified interface for
						learning algorithms in R. 
						\\The \href{https://github.com/mlr-org/mlr3learners}{mlr3learners
							package} and the \href{https://github.com/mlr3learners}{mlr3learners
							organization} \\
						- https://github.com/mlr-org/mlr3learners \\
						- https://github.com/mlr3learners \\
						hold all available learners for {mlr3}.
						\vspace{1em}
						\\
						\begin{codebox}
							learner = \textbf{lrn}(.key, ...)
						\end{codebox}
						Get learner by \codeinline{.key} (from \codeinline{mlr\_learners}) and construct the learner with specific hyperparameter and settings (...) in one go.
						\\
						\vspace{1em}
						\begin{codebox}
							learner\$\textbf{param\_set}
						\end{codebox}
						Returns a description of hyperparameter settings.	
					\end{myblock}
					\vfill
					}
				\end{minipage}
			\end{beamercolorbox}
		\end{column}
		\begin{column}{.245\textwidth}
			\begin{beamercolorbox}[center]{postercolumn}
				\begin{minipage}{.98\textwidth}
					\parbox[t][\columnheight]{\textwidth}{
						\begin{myblock}{Train \& Predict}
						Training: Train a model on a given task.
						\vspace{1em}
						\\
						\begin{codebox}
							learner\$\textbf{train}(task, row\_ids)
						\end{codebox}
						Train a model on the observations in \codeinline{task} given by \codeinline{row\_ids}.
						\\
						\begin{codebox}
							prediction = learner\$\textbf{predict}(task, row\_ids)
						\end{codebox}
						Predict on the observations in \codeinline{task} given by \codeinline{row\_ids}.
						\\
						\vspace{1em}
						\begin{codebox}
							measure = \textbf{msr}(.key)
						\end{codebox}
						Get measure by \codeinline{.key} from \codeinline{mlr\_measures}.
						\\
						\begin{codebox}
							prediction\$\textbf{score}(measure)
						\end{codebox}
						Access performance with \codeinline{measure}.
					\end{myblock}
					\vfill
					}
				\end{minipage}
			\end{beamercolorbox}
		\end{column}
	\end{columns}
\end{frame}
\begin{withoutheader}
\begin{frame}[fragile]{}
	\begin{columns}
		\begin{column}{.245\textwidth}
			\begin{beamercolorbox}[center]{postercolumn}
				\begin{minipage}{.98\textwidth}
					\parbox[t][\columnheight]{\textwidth}{
							\begin{myblock}{Resampling}
							The \codeinline{Resampling} class defines how a task is
							partitioned into a series of train and test sets. \{mlr3\}
							entails six predefined resampling strategies:
							\\
							\begin{itemize}
								\item \codeinline{bootstrap} - Out-of-bag bootstrap.
								\item \codeinline{custom} - Manually provided indices.
								\item \codeinline{cv} - k-fold cross-validation.
								\item \codeinline{holdout} - Holdout-validation.
								\item \codeinline{repeated\_cv} - Repeated k-fold cross-validation.
								\item \codeinline{subsampling} - Monte-Carlo CV.
							\end{itemize}
							\vspace{1em}
							\begin{codebox}
								resampling = \textbf{rsmp}(.key)
							\end{codebox}
							Get resampling strategy by \codeinline{.key}.
							\\
							\begin{codebox}
								resampling\$\textbf{instantiate}(task)
							\end{codebox}
							Apply resampling strategy on \codeinline{task} (perform
							the actual splitting).
						\end{myblock}
						\vfill
					}
				\end{minipage}
			\end{beamercolorbox}
		\end{column}
		\begin{column}{.245\textwidth}
			\begin{beamercolorbox}[center]{postercolumn}
				\begin{minipage}{.98\textwidth}
					\parbox[t][\columnheight]{\textwidth}{
						\begin{myblock}{Resample}
							The \codeinline{resample} function uses the series
							of train and test set(s) defined by the
							\codeinline{Resampling} object to estimate the
							performance of a learning algorithm. For this, a
							model is fitted on each train set
							(\codeinline{\$train()}) and the score is evaluated
							on the corresponding test set(s)
							(\codeinline{\$predict()}) using the given
							\codeinline{Measure}.
							\\
							\\
							\textbf{Careful: Resample $!=$ Resampling}
							\vspace{1em}
							\\
							\begin{codebox}
								rr = \textbf{resample}(task, learner, resampling)
							\end{codebox}
							Performs resampling and returns a \codeinline{ResampleResult} object.
							\\
							\vspace{1em}
							\begin{codebox}
								rr\$\textbf{aggregate}(measure)
							\end{codebox}
							Returns the aggregated performance using the given \codeinline{measure}.
							\\
							\begin{codebox}
								rr\$\textbf{score}(measure)
							\end{codebox}
							Returns the performances of the individual iterations using the given \codeinline{measure}.
						\end{myblock}
						\vfill
					}
				\end{minipage}
			\end{beamercolorbox}
		\end{column}
		\begin{column}{.245\textwidth}
			\begin{beamercolorbox}[center]{postercolumn}
				\begin{minipage}{.98\textwidth}
					\parbox[t][\columnheight]{\textwidth}{
						\begin{myblock}{Benchmark}
						The \codeinline{benchmark} function is used to compare different learners on multiple tasks and/or different resampling schemes.
						\\
						\begin{codeboxmultiline}[width=21.95cm]
							design = \textbf{benchmark\_grid}(\\
							\hspace*{1ex}task = tsks(.keys),\\
							\hspace*{1ex}learner = lrns(.keys),\\
							\hspace*{1ex}resampling = rsmps(.keys)\\
							)
						\end{codeboxmultiline}
						\begin{codebox}
							bmr = \textbf{benchmark}(design)
						\end{codebox}
						\codeinline{benchmark()} executes the benchmark call
						using the matrix given via \codeinline{design} and
						returns a \codeinline{\robotomono{BenckmarkResult}}
						container.
						\begin{codeboxmultiline}[width=27cm]
							\scriptsize{
								<BenchmarkResult> of 404 rows with 8 resampling runs\\
								nr task\_id \space\space\space\space learner\_id resampling\_id iters warnings errors\\
								1 \space\space\space sonar \space classif.rpart
								\space\space\space\space\space\space holdout
								\space\space\space\space 1
								\space\space\space\space\space\space\space 0
								\space\space\space\space\space 0\\
								2 \space\space\space sonar \space classif.rpart
								\space\space repeated\_cv
								\space\space 100
								\space\space\space\space\space\space\space 0
								\space\space\space\space\space 0\\
								3 \space\space\space sonar classif.ranger
								\space\space\space\space\space\space holdout
								\space\space\space\space 1
								\space\space\space\space\space\space\space 0
								\space\space\space\space\space 0\\
								4 \space\space\space sonar classif.ranger
								\space\space repeated\_cv
								\space\space 100
								\space\space\space\space\space\space\space 0
								\space\space\space\space\space 0\\
								5 \space\space\space\space spam
								\space classif.rpart
								\space\space\space\space\space\space holdout
								\space\space\space\space 1
								\space\space\space\space\space\space\space 0
								\space\space\space\space\space 0\\
								6 \space\space\space\space spam
								\space  classif.rpart
								\space\space repeated\_cv
								\space\space 100
								\space\space\space\space\space\space\space 0
								\space\space\space\space\space 0\\
								7 \space\space\space\space spam
								classif.ranger
								\space\space\space\space\space\space holdout
								\space\space\space\space 1
								\space\space\space\space\space\space\space 0
								\space\space\space\space\space 0\\
								8 \space\space\space\space spam
								classif.ranger
								\space\space repeated\_cv
								\space\space 100
								\space\space\space\space\space\space\space 0
								\space\space\space\space\space 0
							}
						\end{codeboxmultiline}
					\end{myblock}
					\vfill
					}
				\end{minipage}
			\end{beamercolorbox}
		\end{column}
		\begin{column}{.245\textwidth}
			\begin{beamercolorbox}[center]{postercolumn}
				\begin{minipage}{.98\textwidth}
					\parbox[t][\columnheight]{\textwidth}{
						\begin{myblock}{Technical Background}
							Content
						\end{myblock}
						\begin{myblock}{Resources}
							\begin{itemize}
								\item \href{https://mlr3book.mlr-org.com/index.html}{mlr3book} (https://mlr3book.mlr-org.com)
								\item \href{https://github.com/mlr-org}{mlr-org on GitHub} (https://github.com/mlr-org)
							\end{itemize}
						\end{myblock}
						\vfill
					}
				\end{minipage}
			\end{beamercolorbox}
		\end{column}
	\end{columns}
\end{frame}
\end{withoutheader}
\end{document}
