\documentclass{beamer}

\usepackage[orientation=landscape,size=a0,scale=1.4,debug]{beamerposter}
\mode<presentation>{\usetheme{mlr}}

\usepackage[sfdefault]{roboto}
\usepackage{roboto-mono}
\usepackage[T1]{fontenc}
\usepackage[utf8]{inputenc} % UTF-8
\usepackage[english]{babel} % Language
\usepackage{hyperref} % Hyperlinks
\usepackage{ragged2e} % Text position
\usepackage[export]{adjustbox} % Image position
\usepackage[most]{tcolorbox} % Code boxes

\hypersetup{
    hyperfootnotes=false,
    colorlinks=true,
	linktocpage=true,
	pdfauthor={mlr-org team},
    %linkcolor=[RGB]{3,99,142}, % mlr blue
    urlcolor=[RGB]{231,138,69}
}

\title{Machine learning with mlr3 :\,: CHEAT SHEET} % Package title in header, \, adds thin space between ::

\newlength{\columnheight} % Adjust depending on header height
\setlength{\columnheight}{84cm}

\newtcolorbox{codebox}{%
	sharp corners,
	leftrule=0pt,
	rightrule=0pt,
	toprule=0pt,
	bottomrule=0pt,
	fontupper=\robotomono\small,
	hbox}

\newtcolorbox{codeboxmultiline}[1][]{%
	sharp corners,
	leftrule=0pt,
	rightrule=0pt,
	toprule=0pt,
	bottomrule=0pt,
	fontupper=\robotomono\small,
	#1}

\newtcolorbox{codeboxexample}{%
	sharp corners,
	leftrule=0pt,
	rightrule=0pt,
	toprule=0pt,
	bottomrule=0pt,
	fontupper=\robotomono\small,
	width=27cm,
	adjusted title=Example,
	fonttitle = \bfseries\Large}

\newtcolorbox{codeboxinline}{%
	sharp corners,
	leftrule=0pt,
	rightrule=0pt,
	toprule=0pt,
	bottomrule=0pt,
	hbox,
	nobeforeafter,
	fontupper=\robotomono\small,
	tcbox raise base}

\newcommand{\codeinline}[1]{\begin{codeboxinline}#1\end{codeboxinline}}

\begin{document}
\begin{frame}[fragile]{}
	\begin{columns}
		\begin{column}{.245\textwidth}
			\begin{beamercolorbox}[center]{postercolumn}
				\begin{minipage}{.98\textwidth}
					\parbox[t][\columnheight]{\textwidth}{
						\begin{myblock}{Intro}
							The \textbf{mlr3} package builds on R6 classes and provides the essential building
							blocks of a machine learning workflow.
						\end{myblock}
						\begin{myblock}{mlr3 Dictionaries}
                            Key-value store for sets of mlr objects. These are provided by mlr3: 
							\\
							\begin{itemize}
								\item \codeinline{mlr\_task} - ML example tasks.
								\item \codeinline{mlr\_task\_generators} - Example generators.
                                \item \codeinline{mlr\_learners} - ML algorithms. 
								\item \codeinline{mlr\_measures} - Performance measures.
								\item \codeinline{mlr\_resamplings} - Resampling strategies.
							\end{itemize}
                            They can be extended and other dicts can be added by extension packages.
                            \codeinline{mlr\_learners} is populated by loading \textbf{mlr3learners} and other learner packages.
							\vspace{1em}		
							\begin{codebox}
								Dictionary\$\textbf{keys}(pattern = NULL)
							\end{codebox}
							Returns all keys which match \codeinline{pattern}. 
							If \codeinline{NULL}, all keys are returned. 
							\\
							\begin{codebox}
								Dictionary\$\textbf{get}(key, ...)
							\end{codebox}
							Retrieves object by \codeinline{key} and 
							passes arguments \codeinline{...} to the construction of the objects.\\
							\begin{codebox}
								Dictionary\$\textbf{mget}(keys, ...)
							\end{codebox}
							Retrieves objects by \codeinline{keys} and 
							passes named arguments \codeinline{...} to the construction of the objects. 
						\end{myblock}
					}
				\end{minipage}
			\end{beamercolorbox}
		\end{column}
		\begin{column}{.245\textwidth}
			\begin{beamercolorbox}[center]{postercolumn}
				\begin{minipage}{.98\textwidth}
					\parbox[t][\columnheight]{\textwidth}{
					\begin{myblock}{Task Class}
						Stores observation and columns (\codeinline{backend}) and additional
						meta-data about the dataset.
						\\
						\begin{codebox}
							task = \textbf{TaskClassif}\$new(backend, target)
						\end{codebox}
						For classification, \codeinline{target} needs to be a character vector representing class labels.
						\\
						\begin{codebox}
							task\$\textbf{positive} = <positive\_class>
						\end{codebox}
						Set positive class for binary classification.
						\\
						\begin{codebox}
							task = \textbf{TaskRegr}\$new(backend, target)
						\end{codebox}
						For regression, \codeinline{target} needs to be a numeric vector.
						\\
                        Get example tasks with \codeinline{tsk(.key)}.
						\vspace{1em}
						\\
						Column roles affect the behavior of the task for different operations. Set with \codeinline{task\$\textbf{col\_roles}\$<role> = <column\_name>}
						\\
						\begin{itemize}
							\item \codeinline{feature} - Regular features
							\item \codeinline{target} - Target variable
							\item \codeinline{name} - Labels for plots
							\item \codeinline{group} -  Groups for resampling
							\item \codeinline{stratum} - Stratification variables
							\item \codeinline{weight} - Observation weights
						\end{itemize}
						\vspace{1em}
						\begin{codebox}
							task\$\textbf{select}(cols)
						\end{codebox}
						Subsets the task based on feature names.
						\\
						\begin{codebox}
							task\$\textbf{filter}(rows)
						\end{codebox}
						Subsets the task based on row ids.
						\\
						\begin{codeboxexample}
							{\footnotesize
							task = tsk("sonar")\\
							learner = lrn("classif.rpart")
							\vspace{1em}
							\\
							train\_set = sample(task\$nrow, 0.8 * task\$nrow)\\
							test\_set = setdiff(seq\_len(task\$nrow), train\_set)
							\vspace{1em}
							\\
							learner\$train(task, row\_ids = train\_set)
							\vspace{1em}
							\\
							prediction = learner\$predict(task,\\
							\hspace*{1ex} row\_ids = test\_set)\\
							prediction\$score()\\
							\#\# classif.ce\\
							\#\# \space 0.2619048}
						\end{codeboxexample}
					\end{myblock}
					\vfill
					}
				\end{minipage}
			\end{beamercolorbox}
		\end{column}
		\begin{column}{.245\textwidth}
			\begin{beamercolorbox}[center]{postercolumn}
				\begin{minipage}{.98\textwidth}
					\parbox[t][\columnheight]{\textwidth}{
						\begin{myblock}{Learner Class}
						Wraps learners from R with a unified interface.
						\\
						\begin{codebox}
							learner = \textbf{lrn}(.key, ...)
						\end{codebox}
						Get learner by \codeinline{.key} (from \codeinline{mlr\_learners}) and construct the learner with specific hyperparameters and settings (...) in one go.
						\\
						\vspace{1em}
						\begin{codebox}
							learner\$\textbf{param\_set}
						\end{codebox}
						Returns description of hyperparameters.	
						\\
						\begin{codebox}
							learner\$param\_set\$\textbf{values} = list(id = value)
						\end{codebox}
						Change the current hyperparameter values by assigning a named \codeinline{list(id = value)} to the \codeinline{\$values} field.
						This overwrites all previously set parameters.	
						\\
						\begin{codeboxmultiline}[width=18cm]
							learner\$param\_set\$\textbf{values} = \\
							\hspace*{1ex} \textbf{mlr3misc::insert\_named}(\\
							\hspace*{2ex} learner\$param\_set\$values,
							\\
							\hspace*{2ex} list(id = value))
						\end{codeboxmultiline}
						Update hyperparameters with named \codeinline{list(id = value)}.
						\vspace{1em}
						\\
						\begin{codebox}
							learner\$\textbf{predict\_type} = "<type>"
						\end{codebox}
						Changes what is technically output during prediction. For classification, 
                        \codeinline{"response"} means class labels, \codeinline{"prob"} means posterior probabilities. For regression, \codeinline{"response"} means numeric response, 
                        \codeinline{"se"} extracts the standard error 
						\vspace{1em}
						% \\
						% Additional learning algorithms are available in the mlr3learners package and the mlr3learners organization (s. Resources).
					\end{myblock}
					\vfill
					}
				\end{minipage}
			\end{beamercolorbox}
		\end{column}
		\begin{column}{.245\textwidth}
			\begin{beamercolorbox}[center]{postercolumn}
				\begin{minipage}{.98\textwidth}
					\parbox[t][\columnheight]{\textwidth}{
						\begin{myblock}{Train \& Predict}
						\begin{codebox}
							learner\$\textbf{train}(task, row\_ids)
						\end{codebox}
                        Train on (selected) observations. 
                        % Model is stored / changed in-place in learner.
						\\
						\begin{codebox}
							learner\$\textbf{model}
						\end{codebox}
						Retrieves the fitted R model.
						\\
						\vspace{1em} % Group Predict
						\begin{codebox}
							prediction = learner\$\textbf{predict}(task, row\_ids)
						\end{codebox}
                        Predict (selected) observations.
						\\
						\begin{codebox}
							measure = \textbf{msr}(.key)
						\end{codebox}
						Get measure by \codeinline{.key} from \codeinline{mlr\_measures}.
						\\
						\begin{codebox}
							prediction\$\textbf{score}(measure)
						\end{codebox}
						Access performance with \codeinline{measure}.
						\\
						\begin{codebox}
							prediction\$\textbf{data\$tab}
						\end{codebox}
						Returns predictions as \codeinline{data.table}.
					\end{myblock}
					\vfill
					}
				\end{minipage}
			\end{beamercolorbox}
		\end{column}
	\end{columns}
\end{frame}
\begin{withoutheader}
\begin{frame}[fragile]{}
	\begin{columns}
		\begin{column}{.245\textwidth}
			\begin{beamercolorbox}[center]{postercolumn}
				\begin{minipage}{.98\textwidth}
					\parbox[t][\columnheight]{\textwidth}{
							\begin{myblock}{Resampling Class}
							Define later partitioning of task into series of train and test sets. \\ 
							Create with \codeinline{resampling = \textbf{rsmp}(.key, ...)}
							\\
							\begin{itemize}
                                \item \codeinline{holdout}
                                (\codeinline{ratio})\\
                                Holdout-validation.
								\item \codeinline{cv}
								(\codeinline{folds})\\
								k-fold cross-validation.
								\item \codeinline{repeated\_cv}
								(\codeinline{folds}, \codeinline{repeats})\\
								Repeated k-fold cross-validation.
								\item \codeinline{subsampling}
								(\codeinline{repeats}, \codeinline{ratio})\\
								Repeated holdouts.
								\item \codeinline{bootstrap}
								(\codeinline{repeats}, \codeinline{ratio})\\
								Out-of-bag bootstrap.
								\item Custom splits \\
								% Manually provided indices.
							\begin{codeboxmultiline}[width=26cm]
							resampling = rsmp("\textbf{custom}")\\
							resampling\$instantiate(task,\\
							\hspace*{1ex} train = list(c(1:10, 51:60, 101:110)),\\
							\hspace*{1ex} test = list(c(11:20, 61:70, 111:120)))
							\end{codeboxmultiline}
							\end{itemize}
							\vspace{1em}
							\begin{codebox}
								resampling\$\textbf{param\_set}
							\end{codebox}
							Returns a description of parameter settings.
							\\
							\begin{codebox}
								resampling\$param\_set\$\textbf{values}(folds = 10)
							\end{codebox}
							Sets folds to 10.
							\\
							\begin{codebox}
								resampling\$\textbf{instantiate}(task)
							\end{codebox}
                            Perform splitting and define index sets. \codeinline{}  will later 
							\\
						\end{myblock}
						\vfill
					}
				\end{minipage}
			\end{beamercolorbox}
		\end{column}
		\begin{column}{.245\textwidth}
			\begin{beamercolorbox}[center]{postercolumn}
				\begin{minipage}{.98\textwidth}
					\parbox[t][\columnheight]{\textwidth}{
						\begin{myblock}{Resample}
							% Uses the \codeinline{Resampling} to estimate the
							% performance of a learning algorithm. 
                            Train-Predict-Score learner for each train/test set.
							% (\codeinline{\$train()}) and the score is evaluated
							% on the corresponding test set(s)
							% (\codeinline{\$predict()}) using the given
							% \codeinline{Measure}.
							% \\
							% \textbf{Careful: Resample $!=$ Resampling}
							% \vspace{1em}
							\\
							\begin{codebox}
								rr = \textbf{resample}(task, learner, resampling)
							\end{codebox}
							Returns a \codeinline{ResampleResult} container object.
							\\
							\vspace{1em}
							\begin{codebox}
								rr\$\textbf{score}(measures)
							\end{codebox}
							Returns datatable of scores on test sets.
							\\
							\begin{codebox}
								rr\$\textbf{aggregate}(measures)
							\end{codebox}
						    Get aggregated performance scores as vector.
							\\
							\begin{codeboxexample}
								\footnotesize{
									task = tsk("pima")\\
									learner = lrn("classif.rpart", \\
									\hspace*{1ex}predict\_type = "prob")\\
									measure = msr("classif.ce")\\
									\vspace{1em}
									\\
									resampling = rsmp("cv", folds = 3L)\\
									resampling\$instantiate(task)\\
									\vspace{1em}
									\\
									rr = resample(task, learner, resampling)\\
									\vspace{1em}
									\\
									rr\$aggregate(measure)\\
									\#\# classif.ce\\
									\#\# \space\space\space\space 0.2643\\
								}
							\end{codeboxexample}
							\end{myblock}
					\vfill}
				\end{minipage}
			\end{beamercolorbox}
		\end{column}
		\begin{column}{.245\textwidth}
			\begin{beamercolorbox}[center]{postercolumn}
				\begin{minipage}{.98\textwidth}
					\parbox[t][\columnheight]{\textwidth}{
						\begin{myblock}{Benchmark}
                            Compare learner(s) on task(s) with resampling(s).
						\\
						\begin{codeboxmultiline}[width=19.4cm]
							design = \textbf{benchmark\_grid}(\\
							\hspace*{1ex}tasks, learners, resamplings)
						\end{codeboxmultiline}
						Creates a cross-join datatable with list-cols, you can also manually define this for full control.
                        % ss joining tasks, learners and resamplings supplied as \codeinline{list()}s.
						\\
						\begin{codebox}
							bmr = \textbf{benchmark}(design)
						\end{codebox}
						Returns a \codeinline{\robotomono{BenckmarkResult}}
						container.\\
						\begin{codeboxmultiline}[width=27cm]
							\scriptsize{
								<BenchmarkResult> of 404 rows with 8 resampling runs\\
								nr task\_id \space\space\space\space learner\_id resampling\_id iters warnings errors\\
								1 \space\space\space sonar \space classif.rpart
								\space\space\space\space\space\space holdout
								\space\space\space\space 1
								\space\space\space\space\space\space\space 0
								\space\space\space\space\space 0\\
								2 \space\space\space sonar \space classif.rpart
								\space\space repeated\_cv
								\space\space 100
								\space\space\space\space\space\space\space 0
								\space\space\space\space\space 0\\
								3 \space\space\space sonar classif.ranger
								\space\space\space\space\space\space holdout
								\space\space\space\space 1
								\space\space\space\space\space\space\space 0
								\space\space\space\space\space 0\\
								4 \space\space\space sonar classif.ranger
								\space\space repeated\_cv
								\space\space 100
								\space\space\space\space\space\space\space 0
								\space\space\space\space\space 0\\
								5 \space\space\space\space spam
								\space classif.rpart
								\space\space\space\space\space\space holdout
								\space\space\space\space 1
								\space\space\space\space\space\space\space 0
								\space\space\space\space\space 0\\
								6 \space\space\space\space spam
								\space  classif.rpart
								\space\space repeated\_cv
								\space\space 100
								\space\space\space\space\space\space\space 0
								\space\space\space\space\space 0\\
								7 \space\space\space\space spam
								classif.ranger
								\space\space\space\space\space\space holdout
								\space\space\space\space 1
								\space\space\space\space\space\space\space 0
								\space\space\space\space\space 0\\
								8 \space\space\space\space spam
								classif.ranger
								\space\space repeated\_cv
								\space\space 100
								\space\space\space\space\space\space\space 0
								\space\space\space\space\space 0
							}
						\end{codeboxmultiline}
						\vspace{1em}
						\begin{codebox}
							bmr\$\textbf{aggregate}(measures)
						\end{codebox}
						Datatable of ResampleResults with scores.
					\end{myblock}
					\begin{myblock}{Parallelization}
						\codeinline{future} is used as backend for parallelization.
                        % e.g. the parallelization of \codeinline{Resampling} iterations.
						\\
						\begin{codebox}
							future::\textbf{plan}("multiprocess")
						\end{codebox}
						Selects the \codeinline{multiprocess} backend. The \codeinline{resample} function is automatically executed in parallel.
					\end{myblock}
					\begin{myblock}{mlr3viz}
						Provides visualization for mlr3 objects.
						Create with \codeinline{mlr3viz::autoplot(object, type)}
						\\
						\begin{itemize}
							\item \codeinline{BenchmarkResult} (\codeinline{boxplot}, \codeinline{roc}, \codeinline{prc})
							\item \codeinline{Filter} (\codeinline{barplot})
							\item \codeinline{PredictionClassif} (\codeinline{stacked}, \codeinline{roc}, \codeinline{prc})
							\item \codeinline{PredictionRegr} (\codeinline{xy}, \codeinline{histogram})
							\item \codeinline{ResampleResult} (\codeinline{boxplot}, \codeinline{histogram}, \codeinline{roc}, \codeinline{prc})
							\item \codeinline{TaskClassif} (\codeinline{target}, \codeinline{duo}, \codeinline{pairs})
							\item \codeinline{TaskRegr} (\codeinline{target}, \codeinline{pairs})
							\item \codeinline{TaskSurv} (\codeinline{target}, \codeinline{duo}, \codeinline{pairs})
						\end{itemize}
					\end{myblock}
					\vfill
					}
				\end{minipage}
			\end{beamercolorbox}
		\end{column}
		\begin{column}{.245\textwidth}
			\begin{beamercolorbox}[center]{postercolumn}
				\begin{minipage}{.98\textwidth}
					\parbox[t][\columnheight]{\textwidth}{
						\begin{myblock}{Logging}
							\codeinline{lgr} is used for logging and progress output.
							\\
							\begin{codeboxmultiline}[width=23.1cm]
								\textbf{getOption}("lgr.log\_levels")\\
								\#\# fatal error  warn  info debug trace\\ 
								\#\# 100 \space\space 200 \space\space 300 \space 400 \space 500 \space\space 600
							\end{codeboxmultiline}
							Gets threshold levels. The default is 400. 
							\\
							\begin{codeboxmultiline}[width=25cm]
								\footnotesize{
								lgr::get\_logger("mlr3")\$\textbf{set\_threshold}("<level>")
							}
							\end{codeboxmultiline}
							Change log-level, you can also do this for other mlr3 packages analogously.
						\end{myblock}
						\begin{myblock}{Error Handling and Encapsulation}
							The \codeinline{evaluate} package is used to encapsulate execution of \codeinline{\$train} and \codeinline{\$predict} to cope with errors, warnings and crashes.
							\\
							\begin{codeboxmultiline}[width=16cm]
								learner\$\textbf{encapsulate} = c(\\
								\hspace*{1ex} train = "evaluate", \\
								\hspace*{1ex} predict = "evaluate")
							\end{codeboxmultiline}
							% Encapsulates the execution of \codeinline{\$train} and \codeinline{\$predict} with the \codeinline{evaluate} package.
							% \\
							\begin{codebox}
								learner\$\textbf{errors}
							\end{codebox}
							Access the recorded log of errors.
                            % via the field \codeinline{log}, \codeinline{warnings} and \codeinline{errors}. 
							\vspace{1em}
							% \\
							% Fallback learners allow to score results in cases where a learners is misbehaving in some sense.
							\\
							\begin{codebox}
								learner\$\textbf{fallback} = lrn(.key)
							\end{codebox}
							If learner fails, fallback is used to generate predictions. 
                            Use robust fallback, e.g. "featureless"! Can ease pain in complex benchmarks when learners fail in edge-cases.
                            % k to the predictions of the attached fallback learner.
						\end{myblock}
						\begin{myblock}{Resources}
							\begin{itemize}
								\item \href{https://mlr3book.mlr-org.com/index.html}{mlr3book}\\ (https://mlr3book.mlr-org.com)
								\item \href{https://github.com/mlr-org}{mlr-org on GitHub}\\ (https://github.com/mlr-org)
								\item \href{https://github.com/mlr-org/mlr3learners}{mlr3learners package}\\ (https://github.com/mlr-org/mlr3learners)
								\item \href{https://github.com/mlr3learners}{mlr3learners organization}\\ (https://github.com/mlr3learners)
							\end{itemize}
						\end{myblock}
						\vfill
					}
				\end{minipage}
			\end{beamercolorbox}
		\end{column}
	\end{columns}
\end{frame}
\end{withoutheader}
\end{document}
