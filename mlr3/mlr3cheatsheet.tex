\documentclass{beamer}

\usepackage[orientation=landscape,size=a0,scale=1.4,debug]{beamerposter}
\mode<presentation>{\usetheme{mlr}}

\usepackage[utf8]{inputenc} % UTF-8
\usepackage[english]{babel} % Language
\usepackage{hyperref} % Hyperlinks
\usepackage{ragged2e} % Text position
\usepackage[export]{adjustbox} % Image position
\usepackage[most]{tcolorbox}
\usepackage{listings} % for R code
\lstset{language=R,
    basicstyle=\small\ttfamily,
    stringstyle=\color{DarkGreen},
    otherkeywords={0,1,2,3,4,5,6,7,8,9},
    morekeywords={TRUE,FALSE},
    deletekeywords={data,frame,length,as,character},
    keywordstyle=\color{blue},
    commentstyle=\color{DarkGreen},
}

\title{mlr3 :\,: CHEAT SHEET} % Package title in header, \, adds thin space between ::
\newcommand{\packagedescription}{ % Package description in header
	The \textbf{mlr3} package provides a framework for classification, regression and other machine learning tasks.
}

\newlength{\columnheight} % Adjust depending on header height
\setlength{\columnheight}{84cm} 

\newtcolorbox{codebox}{%
	sharp corners,
	leftrule=0pt,
	rightrule=0pt,
	toprule=0pt,
	bottomrule=0pt,
	hbox}

\newtcolorbox{codeboxmultiline}[1][]{%
	sharp corners,
	leftrule=0pt,
	rightrule=0pt,
	toprule=0pt,
	bottomrule=0pt,
	#1}

\begin{document}
\begin{frame}[fragile]{}
\begin{columns}
	\begin{column}{.245\textwidth}
		\begin{beamercolorbox}[center]{postercolumn}
			\begin{minipage}{.98\textwidth}
				\parbox[t][\columnheight]{\textwidth}{
					\begin{myblock}{Resources}
						\begin{itemize}
							\item \href{https://mlr3book.mlr-org.com/index.html}{mlr3book}
							\item \href{https://github.com/mlr-org}{mlr-org on GitHub} 
						\end{itemize}
					\end{myblock}
						\begin{myblock}{Packages}
										\vfill
						\begin{itemize}
							\item \href{https://github.com/mlr-org/mlr3pipelines}{mlr3pipelines} - Dataflow Programming
							\item \href{https://github.com/mlr-org/mlr3tuning}{mlr3tuning} - Tuning Methods
							\item \href{https://github.com/mlr-org/mlr3filters}{mlr3filters} - Feature Selection Filters
							\item \href{https://github.com/mlr-org/mlr3survival}{mlr3survival} - Survival analysis
							\item \href{https://github.com/mlr-org/mlr3fswrap}{mlr3fswrap} - Wrapper feature selection
							\item \href{https://github.com/mlr-org/mlr3learners}{mlr3learners} - Recommended learners
							\item \href{https://github.com/mlr-org/mlr3viz}{mlr3viz} - Visualizations
							\item \href{https://github.com/mlr-org/mlr3ordinal}{mlr3ordinal} - Ordinal Regression
						\end{itemize}
					\end{myblock}
					\begin{myblock}{Intro \& Workflow}
					The mlr3 package provides R6 classes for the essential building
					blocks of this machine learning workflow:
					\begin{enumerate}
						\item A \textbf{task} encapsulates the data along with additional information, such as 
						what the prediction target is.
						\item A \textbf{learner} encapsulates one of R's many machine learning algorithms and allows to train models and make predictions. Most learners have hyperparameters.
						\item A \textbf{measure} computes a numeric score based on predicted and ground-truth values and their difference.
						\item A \textbf{resampling} specifies a series of train and test sets.
					\end{enumerate}
				\includegraphics[width=\textwidth]{img/ml_abstraction.png}
          		\end{myblock}
				}
			\end{minipage}
		\end{beamercolorbox}
	\end{column}
	\begin{column}{.245\textwidth}
		\begin{beamercolorbox}[center]{postercolumn}
			\begin{minipage}{.98\textwidth}
				\parbox[t][\columnheight]{\textwidth}{
				  \begin{myblock}{Task}
						Tasks objects store data (\textit{backend}) and additional meta-data for machine learning problems. A \textit{backend} allows fine-grained control over how data is accessed. If you simply provide the dataset, it is automatically converted to a \textit{DataBackendDataTable}.
						\\
						\begin{codebox}
							task = \textbf{TaskClassif}\$new(backend, target)
						\end{codebox}
						The \textit{target} is handed over as character value. For classification, the name of the target column is a label with only few distinct values.
						\\
						\begin{codebox}
							task = \textbf{TaskRegr}\$newt(backend, target)
						\end{codebox}
						The \textit{target} is a numeric quantity.
						\\
						\begin{codebox}
							task = \textbf{mlr\_tasks}\$\textbf{get}(key)
						\end{codebox}
						Get predefined task by \textit{key} from mlr\_task dictionary. The key (character value) identifies these tasks. 
						\\
						\begin{codebox}
							task = \textbf{tsk}(key)
						\end{codebox}
						Get predefined task by \textit{key} from mlr\_task dictionary with convenience function.
						\\
						\begin{codebox}
							task\$\textbf{data}()
						\end{codebox}
						Retrieves stored data.
						\\
						\begin{codebox}
							task\$\textbf{set\_col\_role}(cols, new\_roles)
						\end{codebox}
						\begin{codebox}
							task\$\textbf{set\_row\_role}(rows, new\_roles)
						\end{codebox}
						 Assign role (\textit{new\_roles}) to \textit{cols} / \textit{rows}. These task methods change the view on the data and can be used to subset the task.
						\\
						\begin{codebox}
							task\$\textbf{select}(cols)
						\end{codebox}
						Subsets the task based on feature names (\textit{cols}).
						\\
						\begin{codebox}
							task\$\textbf{filter}(rows)
						\end{codebox}
						Subsets the task based on row ids (\textit{rows}).
					\end{myblock}
					\vfill
				}
			\end{minipage}
		\end{beamercolorbox}
	\end{column}
	\begin{column}{.245\textwidth}
		\begin{beamercolorbox}[center]{postercolumn}
			\begin{minipage}{.98\textwidth}
				\parbox[t][\columnheight]{\textwidth}{
					\begin{myblock}{Learner}
						Learner objects provide a unified interface to machine learning algorithms.
						\\
						\begin{codebox}
							learner = \textbf{mlr\_learners}\$\textbf{get}(key)
						\end{codebox}
						 Get learner by \textit{key} from mlr\_learner dictionary.
						\\
						\begin{codebox}
							learner = \textbf{lrn}(key, ...)
						\end{codebox}
						 Get learner by \textit{key} and construct the learner with specific hyperparameter and settings (...) in one go.
						\\
						\begin{codebox}
							learner\$\textbf{param\_set}
						\end{codebox}
						Returns a description of hyperparameter settings.
					\end{myblock}
				\begin{myblock}{Train \& Predict}
					Training fits a model to a given task. 
					\\
					\begin{codebox}
						train\_set = \textbf{sample}(task\$nrow, 0.8 * task\$nrow)
					\end{codebox}
					Get train set.
					\\
					\begin{codebox}
						test\_set = \textbf{setdiff}(seq\_len(task\$nrow), train\_set)
					\end{codebox}
					Get test set.
					\\
					\begin{codebox}
						learner\$\textbf{train}(task, row\_ids = train\_set)
					\end{codebox}
					Train model with the \textit{train\_set}.
					\\
					\begin{codebox}
						prediction = learner\$\textbf{predict}(task, row\_ids = test\_set)
					\end{codebox}
					Predict \textit{test\_set} with model.
					\\
					\begin{codebox}
						measure = \textbf{mlr\_measures}\$\textbf{get}(key)
					\end{codebox}
					Get measure by \textit{key} from mlr\_measure dictionary.
					\\
					\begin{codebox}
						measure = \textbf{msr}(key)
					\end{codebox}
					Get measure by \textit{key} with convenience function.
					\\
					\begin{codebox}
						prediction\$\textbf{score}(measure)
					\end{codebox}
					Access performance with \textit{measure}.
				\end{myblock}
										}
			\end{minipage}
		\end{beamercolorbox}
	\end{column}
		\begin{column}{.245\textwidth}
		  \begin{beamercolorbox}[center]{postercolumn}
			   \begin{minipage}{.98\textwidth}
				  \parbox[t][\columnheight]{\textwidth}{
					  \begin{myblock}{Resampling}
						  Resampling is used to assess the performance of a learning algorithm.
						  \\[\baselineskip]
						    \begin{minipage}{\textwidth}
							    \begin{columns}[T]
								    \begin{column}{0.2\textwidth}\leftskip=14pt
									    \includegraphics[width=\textwidth]{img/cross_validation.png}
								    \end{column}
							    	\begin{column}{0.01\textwidth}
							    	% Space between image and text
							    	\end{column}
								    \begin{column}{0.79\textwidth}
										  \begin{codebox}
										    resampling = \textbf{mlr\_resamplings}\$\textbf{get}(key)
										  \end{codebox}
										  Get resampling strategy by \textit{key} from mlr\_resamplings dictionary.
								    \end{column}
							     \end{columns}
					      	\end{minipage}
						      \\[\baselineskip]
						      \begin{codebox}
						      	resampling = \textbf{rsmp}(key)
						      \end{codebox}
						      Get resampling strategy by \textit{key} with convenience function.
						      \\
						      \begin{codebox}
							      resampling\$\textbf{instantiate}(task)
						      \end{codebox}
						      Apply splitting on \textit{task}.
						      \\
					      	\begin{codebox}
							      rr = \textbf{resample}(task, learner, resampling)
						      \end{codebox}
						      Executes resampling.
						      \\
						      \begin{codebox}
							      rr\$\textbf{performance}(measure)
						      \end{codebox}
						      Extract performance of individual resampling iterations with \textit{measure}.
					      \end{myblock}
				        \begin{myblock}{Benchmarking}
						      Benchmarking is used to compare the performance of different learners on multiple task and/or different resampling schemes.
						    \\
						      \begin{codeboxmultiline}[width=21.95cm]
							      design = \textbf{expand\_grid}(\\
							      \hspace*{1ex}tasks = mlr\_tasks\$mget(key),\\
							      \hspace*{1ex}learners = mlr\_learners\$mget(key),\\
							      \hspace*{1ex}resamplings = mlr\_resamplings\$mget(key))
						      \end{codeboxmultiline}
						      Create benchmark design.
						      \\
						      \begin{codebox}
							      bmr = \textbf{benchmark}(design)
						      \end{codebox}
						      Execute benchmark with \textit{design}.
					      	\\
						      \begin{codebox}
							      bmr\$\textbf{aggregate}(measure)
						      \end{codebox}
						      Calculate and aggregate performance.
					        \end{myblock}\vfill
				            }
		          	\end{minipage}
		          \end{beamercolorbox}
	           \end{column}
            \end{columns}
          \end{frame}
        \end{document}
