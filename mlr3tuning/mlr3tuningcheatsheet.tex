\documentclass{beamer}

\usepackage[orientation=landscape,size=a0,scale=1.4,debug]{beamerposter}
\mode<presentation>{\usetheme{mlr}}

\usepackage[sfdefault]{roboto}
\usepackage{roboto-mono}
\usepackage[T1]{fontenc}
\usepackage[utf8]{inputenc} % UTF-8
\usepackage[english]{babel} % Language
\usepackage{hyperref} % Hyperlinks
\usepackage{ragged2e} % Text position
\usepackage[export]{adjustbox} % Image position
\usepackage[most]{tcolorbox}
\usepackage{listings} % for R code
\lstset{language=R,
    basicstyle=\small\ttfamily,
    stringstyle=\color{DarkGreen},
    otherkeywords={0,1,2,3,4,5,6,7,8,9},
    morekeywords={TRUE,FALSE},
    deletekeywords={data,frame,length,as,character},
    keywordstyle=\color{blue},
    commentstyle=\color{DarkGreen},
}

\title{mlr3tuning :\,: CHEAT SHEET} % Package title in header, \, adds thin space between ::
\newcommand{\packagedescription}{ % Package description in header
	The \textbf{mlr3tuning} package provides hyperparameter tuning for mlr3.
}

\newlength{\columnheight} % Adjust depending on header height
\setlength{\columnheight}{84cm} 

\newtcolorbox{codebox}{%
	sharp corners,
	leftrule=0pt,
	rightrule=0pt,
	toprule=0pt,
	bottomrule=0pt,
	fontupper=\robotomono\small,
	hbox}

\newtcolorbox{codeboxmultiline}[1][]{%
	sharp corners,
	leftrule=0pt,
	rightrule=0pt,
	toprule=0pt,
	bottomrule=0pt,
	fontupper=\robotomono\small,
	#1}

\newtcolorbox{codeboxinline}{%
	sharp corners,
	leftrule=0pt,
	rightrule=0pt,
	toprule=0pt,
	bottomrule=0pt,
	hbox,
	nobeforeafter,
	fontupper=\robotomono\small,
	tcbox raise base}

\newcommand{\codeinline}[1]{\begin{codeboxinline}#1\end{codeboxinline}}

\begin{document}
\begin{frame}[fragile]{}
	\begin{columns}
		\begin{column}{.245\textwidth}
			\begin{beamercolorbox}[center]{postercolumn}
				\begin{minipage}{.98\textwidth}
					\parbox[t][\columnheight]{\textwidth}{
						\begin{myblock}{Intro}
							The mlr3tuning package is an extension for the \href{https://github.com/mlr-org/mlr3}{mlr3} package and provides R6 classes for hyperparameter tuning.
							\\
							\\
							\includegraphics[width=\textwidth]{img/tuning_objects.png}
							\\
							\\
							The heart of mlr3tuning are the classes \codeinline{TuningInstance} and \codeinline{Tuner}. The processes shown in the figure are referenced in the cheatsheet. 
						\end{myblock}
						\begin{myblock}{ParamterSet}
							The \codeinline{ParamSet} defines the hyperparameter to tune and the tuning space (1).
							\\
							\begin{codeboxmultiline}[width=21cm]
								tune\_ps = \textbf{ParamSet}\$new(list(\\
								\hspace*{1ex}\textbf{ParamInt}\$new(id, lower, upper),\\
								\hspace*{1ex}\textbf{ParamDbl}\$new(id, lower, upper),\\
								\hspace*{1ex}\textbf{ParamFct}\$new(id, levels),\\
								\hspace*{1ex}\textbf{ParamLgl}\$new(id)))
							\end{codeboxmultiline}
							Constructs \codeinline{ParamSet} with different types of \codeinline{Param}s. Set \textit{id} to one of the hyperparameter ids of the learner. For  \codeinline{ParamInt} and \codeinline{ParamDbl} set \textit{lower} and \textit{upper} to define the tuning space. Use \textit{levels} in \codeinline{ParamFct} to define the factor levels to tune over.
							\\
							\begin{codebox}
								tune\_ps\$\textbf{add}(id, on, cond)
							\end{codebox}
							Adds dependency to the \codeinline{ParamSet}, so that \textit{id} depends on parameter \textit{on}.
							\textit{cond} is an object of class \codeinline{Condition}.
							\\
							\begin{codeboxmultiline}[width=26cm]
								tune\_ps\$\textbf{trafo} = function(x, param\_set) \{ \\
								\hspace*{1ex}x\$id = -log(x\$id)\}
							\end{codeboxmultiline}
							Set a transformation for parameter \textit{id}.
						\end{myblock}
						\vfill}
				\end{minipage}
			\end{beamercolorbox}
		\end{column}
		\begin{column}{.245\textwidth}
			\begin{beamercolorbox}[center]{postercolumn}
				\begin{minipage}{.98\textwidth}
					\parbox[t][\columnheight]{\textwidth}{
						\begin{myblock}{Terminator}
							The \codeinline{Terminator} determines when to stop the tuning (2). The package provides four \codeinline{Terminator} classes:
							\\
							\begin{itemize}
								\item \codeinline{clock\_time} - After a given time.
								\item \codeinline{evals} - After a given amount of iterations.
								\item \codeinline{model\_time}  - After a given model time.
								\item \codeinline{perf\_reached} - After a specific performance.
								\item \codeinline{stagnation} - After the performance stagnates.
							\end{itemize}
							\vspace{0.5cm}
							Keys to access the \codeinline{Terminator} subclasses.
							\\
							\begin{codebox}
								terminator = \textbf{term}(.key, ...)
							\end{codebox}
							Get terminator by \textit{.key} and construct terminator with settings (...) in one go.
							\\
							\begin{codebox}
								terminator = term("\textbf{combo}", terminators, any)
							\end{codebox}
							List of \textit{terminators} that terminate if any (\codeinline{any = TRUE}) or all (\codeinline{any = FALSE}) terminators are positive.
						\end{myblock}
						\begin{myblock}{TuningInstance}
							The \codeinline{TuningInstance} specifies a general search scenario. It evaluates the hyperparameter configurations proposed by the \codeinline{Tuner} and stores the results (s. Triggering the tuning).
							\\
							\begin{codeboxmultiline}[width=20cm]
								instance = \textbf{TuningInstance}\$new(\\
								\hspace*{1ex}task,\\
								\hspace*{1ex}learner,\\
								\hspace*{1ex}resampling,\\
								\hspace*{1ex}tune\_ps,\\
								\hspace*{1ex}terminator)
							\end{codeboxmultiline}
							Constructs the \codeinline{TuningInstance} (3).
							\\
						\end{myblock}
						\vfill}
				\end{minipage}
			\end{beamercolorbox}
		\end{column}
		\begin{column}{.245\textwidth}
			\begin{beamercolorbox}[center]{postercolumn}
				\begin{minipage}{.98\textwidth}
					\parbox[t][\columnheight]{\textwidth}{
						\begin{myblock}{Tuner}
							The \codeinline{Tuner} describes the tuning strategy. The package provides three tuning strategies and additionally hyperparameter configurations fully specified by the user:
							\\
							\begin{itemize}
								\item \codeinline{grid\_search} - Grid search.
								\item \codeinline{random\_search} - Random search.
								\item \codeinline{gensa} - Generalized simulated annealing.
								\item \codeinline{design\_points} - Specified by the user.
							\end{itemize}
							\vspace{0.5cm}
							Keys to access the \codeinline{Tuner} subclasses.
							\\
							\begin{codebox}
								tuner = \textbf{tnr}(.key, ...)
							\end{codebox}
							Get the tuner by \textit{.key} and construct the tuner with settings (...) in one go.
						\end{myblock}
						\begin{myblock}{Triggering the tuning}
							To start the tuning, \codeinline{TuningInstance} is passed to the \codeinline{tune} method of \codeinline{Tuner}.
							\\
							\begin{codebox}
								tuner\$\textbf{tune}(instance)
							\end{codebox}
							Starts the tuning (4).
							\\
							\\
							The \codeinline{Tuner} generates hyperparameter configurations and passes them to \codeinline{TuningInstance} until the budget of \codeinline{Terminator} is exhausted (5). To access the results use the following methods of \codeinline{TuningInstance}:
							\\
							\begin{codebox}
								instance\$\textbf{archive}(unnest)
							\end{codebox}
							Returns all resampling results together with the used hyperparameters. Use \textit{unnest} to display hyperparameter without (\textit{tune\_x}) or with (\textit{params}) trafo applied.
							\\
							\begin{codebox}
								instance\$\textbf{result}
							\end{codebox}
							Returns a list with the optimal hyperparameter configuration and the estimated performance.
						\end{myblock}
						\vfill}
				\end{minipage}
			\end{beamercolorbox}
		\end{column}
		\begin{column}{.245\textwidth}
			\begin{beamercolorbox}[center]{postercolumn}
				\begin{minipage}{.98\textwidth}
					\parbox[t][\columnheight]{\textwidth}{
						\begin{myblock}{Automatic Tuning}
							The \codeinline{AutoTuner} wraps a learner and augments it with an automatic tuning for a given set of hyperparameters.
							\\
							\begin{codeboxmultiline}[width=18cm]
								at = \textbf{AutoTuner}\$new(
								\hspace*{1ex}learner,\\
								\hspace*{1ex}resampling,\\
								\hspace*{1ex}measures,\\
								\hspace*{1ex}tune\_ps,\\
								\hspace*{1ex}terminator,\\
								\hspace*{1ex}tuner)
							\end{codeboxmultiline}
							Constructs the \codeinline{AutoTuner}.
							\\
							\\
							The \codeinline{AutoTuner} inherits from the \codeinline{Learner} class and therefore can be used like any other learner.
							\\
							\begin{codebox}
								at\$\textbf{train}(task)
							\end{codebox}
							The automatic tuning is started by supplying \textit{task} to the \codeinline{train} method.
							\\
							\begin{codebox}
								at\$\textbf{predict}(task, row\_ids)
							\end{codebox}
							Use the tuned \codeinline{Learner} in \codeinline{AutoTuner} to create a new \codeinline{Prediction} by supplying \textit{task} and \textit{row\_ids}.
							\\
						\end{myblock}
						\vfill}
				\end{minipage}
			\end{beamercolorbox}
		\end{column}
	\end{columns}
\end{frame}
\end{document}
